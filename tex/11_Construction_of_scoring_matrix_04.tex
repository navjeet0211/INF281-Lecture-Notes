%\documentclass[12pt]{article}
%\usepackage[a4paper, margin=1in]{geometry} 
%\usepackage{graphicx} 
%\usepackage{hyperref}
%\usepackage{float}
%\usepackage{multicol}
%\usepackage{multirow}
%\usepackage{amsmath}
%\usepackage{colortbl}
%\definecolor{lightgray}{RGB}{211, 211, 211}
%\usepackage[font=small, labelfont=bf]{caption}
%
%\begin{document}

%
% BLOSUM
%
\subsection{BLOSUM}
BLOSUM is another popular method of constructing a scoring marix. It is more useful for diverse sequences than PAM.

%
% Resources of constructing a BLOSUM score
%
\subsubsection*{Resources of constructing a BLOSUM score}
\begin{itemize}
\item Scanned very conserved regions of protein families on the BLOCKS database
\item Identified 2000 blocks
\item A block contains multiple sequences that are highly conserved
\end{itemize}

%
% Observed mutations
%
\subsubsection*{Observed mutations}

\begin{center}
$\begin{aligned}
T : & \text{ The total number of pairs from all blocks.} \\
& \text{ The number of amino acid pairs of a block with length } w \\
& \text{ and } m \text{ sequences can be calculated as } 1/2wm(m-1). \\
f_{ab} : & \text{ The frequencies of an observed pair } a \text{ and } b.\\
\end{aligned} $
\end{center}

%
% Example of observed mutations
%
\subsubsection*{Example of observed mutations}

\begin{table}[H]
\centering
\begin{tabular}{|l|l|}
\hline
Block1 & Block2 \\ \hline
\verb|AGCC|   & \verb|AGA|    \\ \hline
\verb|TAGC|   & \verb|TAC|    \\ \hline
\verb|AGCC|   &        \\ \hline
\end{tabular}
\end{table}

\[
T=1/2 \cdot 4 \cdot 3 \cdot 2+ 1/2 \cdot 3 \cdot 2 \cdot 1 = 12 + 3 = 15
\]

\begin{table}[H]
\centering
\begin{tabular}{|rl|rl|rl|rl|}
\hline
$f_{AA}$: & 1/15  & $f_{AC}$: & 1/15 & $f_{AG}$: & 3/15 & $f_{AT}$: & 3/15 \\ \arrayrulecolor{lightgray} \hline \arrayrulecolor{black} 
$f_{GG}$: & 1/15  & $f_{GC}$: & 2/15 & $f_{CC}$: & 4/15 &  &    \\ \hline
\end{tabular}
\end{table}

%
% Example of observed mutations
%
\subsubsection*{Example of observed mutations}

\begin{itemize}

\item Frequencies of estimated mutations

\begin{table}[H]
\centering
\begin{tabular}{|rl|rl|rl|rl|}
\hline
$f_A$: & 2  & $f_G$: & 2 & $f_C$: & 4 & $f_T$: & 2  \\ \arrayrulecolor{lightgray} \hline \arrayrulecolor{black} 
$f:$ & 10 &   &   &  &   &  &   \\ \hline
\end{tabular}
\end{table}

\end{itemize}

%
% Background frequencies
%
\subsubsection*{Background frequencies}

\begin{center}
$\begin{aligned}
p_a &= f_{aa} + \sum_{e \neq a}{} \dfrac{f_{ae}}{2}\\
e_{aa} &= p_a \cdot p_a\\
e_{ab} &= p_a \cdot p_b + p_b \cdot p_a = 2 \cdot p_a \cdot p_b \\
\end{aligned} $
\end{center}

%
% Example of background frequencies
%
\subsubsection*{Example of background frequencies}

\begin{center}
$\begin{aligned}
p_A &= \dfrac{1}{15}  + \dfrac{1}{2}  \times \left( \dfrac{3}{15}  + \dfrac{1}{15} + \dfrac{3}{15} \right) 
   = \dfrac{1}{15}  + \dfrac{7}{30} 
   = \dfrac{9}{30} \\
p_G &= \dfrac{1}{15}  + \dfrac{1}{2}  \times \left( \dfrac{3}{15}  + \dfrac{2}{15} \right) 
   = \dfrac{1}{15}  + \dfrac{5}{30} 
   = \dfrac{7}{30} \\
p_C &= \dfrac{4}{15}  + \dfrac{1}{2}  \times \left( \dfrac{2}{15}  + \dfrac{1}{15} \right) 
   = \dfrac{4}{15}  + \dfrac{3}{30} 
   = \dfrac{11}{30} \\
p_T &= \dfrac{0}{15}  + \dfrac{1}{2}  \times \left( \dfrac{3}{15}  \right) 
   = \dfrac{0}{15}  + \dfrac{3}{30} 
   = \dfrac{3}{30} \\ \\
e_{AA} &= p_A \cdot p_A = \dfrac{9}{30}  \times \dfrac{9}{30} = \dfrac{81}{900} \\
e_{AG} &= p_A \cdot p_G = 2 \times \dfrac{9}{30}  \times \dfrac{7}{30} = \dfrac{126}{900} \\
\end{aligned} $
\end{center}

%
% Scoring matrix
%
\subsubsection*{Scoring matrix}
BLOSUM scores are calculated as log ratios of observed and background probabilities.

\[
R_{ab} = \log \dfrac{f_{ab}}{e_{ab}}
\]

%
% BLOSUM scores of different distances
%
\subsubsection*{BLOSUM scores of different distances}
One can categorize segments by an identify x to create BLOSUM x.

%
% Example of BLOSUM x
%
\subsubsection*{Example of BLOSUM x}

\begin{table}[H]
\centering
\begin{tabular}{|l|l|}
\hline
1 & \verb|AGCC| \\ \hline
2 & \verb|TAGC| \\ \hline
3 & \verb|AGTC| \\ \hline
4 & \verb|AGTT| \\ \hline
\end{tabular}
\end{table}

100\% identity
\begin{table}[H]
\centering
\begin{tabular}{|l|l|c}
\cline{1-2}
1 & \verb|AGCC| &  \\ \cline{1-2}
2 & \verb|TAGC| & \multirow{2}{*}{Number of mutations in the first column:}  \\ \cline{1-2}
3 & \verb|AGTC| & \multirow{2}{*}{6 (3 ATs \& 3 AAs)}     \\ \cline{1-2}
4 & \verb|AGTT| &                       \\ \cline{1-2}
\end{tabular}
\end{table}

\noindent
75\% identity
\begin{table}[H]
\centering
\begin{tabular}{|l|l|c}
\cline{1-2}
    & \verb|  C|    &                          \\
1,3 & \verb|AG C| &                          \\
    & \verb|  T|    & Number of mutations in the first column: \\ \cline{1-2}
2   & \verb|TAGC| & 3 (2 ATs \& 1 AA)  \\ \cline{1-2}
4   & \verb|AGTT| &                          \\ \cline{1-2}
\end{tabular}
\end{table}

\noindent
50\% identity
\begin{table}[H]
\centering
\begin{tabular}{|l|l|c}
\cline{1-2}
      & \verb|  CC|   &                          \\
1,3,4 & \verb|AGTC| & Number of mutations in the first column:  \\
      & \verb|  TT|   &  1 (1 AT) \\ \cline{1-2}
2     & \verb|TAGC| &                          \\ \cline{1-2}
\end{tabular}
\end{table}

\bigskip 

%\end{document}
