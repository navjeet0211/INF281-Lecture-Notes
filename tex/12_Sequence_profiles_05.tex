%\documentclass[12pt]{article}
%\usepackage[a4paper, margin=1in]{geometry} 
%\usepackage{graphicx} 
%\usepackage{hyperref}
%\usepackage{float}
%\usepackage{multicol}
%\usepackage{multirow}
%\usepackage{amsmath}
%\usepackage[ruled]{algorithm2e}
%\usepackage[font=small, labelfont=bf]{caption}
%
%\begin{document}

%
% PSI-BLAST
%
\subsection{PSI-BLAST}
Position-specific iterated BLAST  (PSI-BLAST) is an extension of BLAST. It is much more sensitive than BLAST. It can be used to find distantly related proteins.

%
% Simplified procedure of PSI-BLAST
%
\subsubsection*{Pseudo-code of linear progressive alignment (general progressive alignment)}

\begin{algorithm}[H]
  \SetKwInOut{Q}{$\mathrm{q}$}
  \SetKwInOut{T}{$\mathrm{t}$}
  \SetKwRepeat{Do}{do}{while}
  
  \BlankLine
    
  \Q{query sequence}
  \T{threshold for significance}
  
  \BlankLine \BlankLine
  
   Q = BLAST(q, t);

  \BlankLine \BlankLine

  \Do{\textsf{\upshape convergence(Reduce(Q) = Q1) or maximum number of cycles}}{
     Q1 = Reduce(Q)               \tcp*{Remove identical segments}
     M  = MultipleAlignment(Q1); \\
     P  = Profile(M); \\
     Q  = ProfileSearch(P);
  }

  \BlankLine   \BlankLine
  
  \SetAlgoRefName{\thesection.1}
  \caption{Simplified procedure of PSI-BLAST}

\end{algorithm}

\bigskip 

%\end{document}
