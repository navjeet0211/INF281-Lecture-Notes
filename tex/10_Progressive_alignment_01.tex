%\documentclass[12pt]{article}
%\usepackage[a4paper, margin=1in]{geometry} 
%\usepackage{graphicx} 
%\usepackage{hyperref}
%\usepackage{float}
%\usepackage{multicol}
%\usepackage{multirow}
%\usepackage{amsmath}
%\usepackage[font=small, labelfont=bf]{caption}
%
%\begin{document}

%
% Introduction to progressive alignment
%
\subsection{Introduction to progressive alignment}
Several heuristic solutions to compute MSAs have been developed to avoid the multi-dimensional DP approach that requires heavy computational power.

%
% Three cases of aligning multiple sequences
%
\subsubsection*{Three cases of aligning multiple sequences}
\begin{itemize}
\item Two sequences, e.g. $s^1$ and $s^2$
\item One alignment and one sequence, e.g. $\mathcal{A}^1$ and $s^1$
\item Two alignments, e.g. $\mathcal{A}^1$ and $\mathcal{A}^2$
\end{itemize}

%
% Guiding methods
%
\subsubsection*{Guiding methods}
\begin{itemize}
\item Clustering
\item Phylogenetic tree
\end{itemize}

%
% Aligning methods
%
\subsubsection*{Aligning methods}
\begin{itemize}
\item Complete alignment
\item Pair-guided alignment
\end{itemize}

%
% Once a gap always a gap
%
\subsubsection*{Once a gap always a gap}
Many progressive alignment procedures use the “once a gap always a gap” policy, hence it is difficult to fix the errors that are made in early steps.

\bigskip 

%\end{document}
