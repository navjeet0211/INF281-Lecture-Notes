%\documentclass[12pt]{article}
%\usepackage[a4paper, margin=1in]{geometry} 
%\usepackage{graphicx} 
%\usepackage{hyperref}
%\usepackage{float}
%\usepackage{multicol}
%\usepackage{multirow}
%\usepackage{amsmath}
%\usepackage[font=small, labelfont=bf]{caption}
%
%\begin{document}

%
% Tree reconstruction methods
%
\subsection{Tree reconstruction methods}
A number of methods have been proposed to reconstruct a phylogenetic tree.

%
% Two types of reconstruction methods
%
\subsubsection*{Two types of reconstruction methods}
\begin{itemize}
\item Distance-based methods
\item Character-based methods
\end{itemize}

%
% Distance-based methods
%
\subsubsection*{Distance-based methods}
A distance is a positive value with larger values indicating that two sequences are separated further. 

\begin{itemize}
\item PGMA (pair-group method using arithmetic mean)
\item Neighbor-joining  (NJ)
\end{itemize}

%
% Character-based methods
%
\subsubsection*{Character-based methods}
Character based methods rely on characters (amino acid/nucleotide letters) to reconstruct a tree.

\begin{itemize}
\item Maximum parsimony
\item Maximum likelihood
\end{itemize}

% Evaluation of reconstructed tree
%
\subsubsection*{Evaluation of reconstructed trees}
Bootstrapping is one of the methods to test the robustness of a reconstruct tree by adding noises and comparing the results.
\begin{enumerate}
\item Randomly generate a pseudo MAS from the original MSA
\item Reconstruct a tree
\item Repeat the process
\item Compare the trees
\end{enumerate}

\bigskip 

%\end{document}
