%\documentclass[12pt]{article}
%\usepackage[a4paper, margin=1in]{geometry} 
%\usepackage{graphicx} 
%\usepackage{hyperref}
%\usepackage{float}
%\usepackage{multicol}
%\usepackage[font=small, labelfont=bf]{caption}
%
%\begin{document}

%
% Alignment by brute-force
%
\subsection{Alignment by brute--force}
A brute--force approach finds the alignment with the highest score by simply considering all possible alignments and calculates the score for each of them.

%
% An example of brute--force approach
%
\subsubsection*{An example of brute--force approach}
We find the optimal alignment for the following sequences by using the scoring scheme below.

\begin{multicols}{2}
Sequences:
\begin{verbatim}
    q: AG, d: ACG
\end{verbatim}
\vfill\null
\columnbreak

\noindent Scoring scheme: \\ 
\null \quad $R_{ab}$ = 1 for a = b \\ 
\null \quad $R_{ab}$ = 0 for a $\neq$ b \\ 
\null \quad g = 1

\end{multicols} 

\noindent \textbf{1. The length of alignment}
\begin{itemize}
\item Maximum length: length(q) + length(d)
\item Minimum length: max(length(q), length(d))
\end{itemize}
\medskip 

\noindent \textbf{2. All possible alignments when length = 5}
\begin{verbatim}
    ---AG    A---G    A--G-    AG---    --A-G
    ACG--    -ACG-    -AC-G    --ACG    AC-G-

    --AG-    -AG--    -A--G    -A-G-    A-G--
    AC--G    A--CG    A-CG-    A-C-G    -A-CG
\end{verbatim}
\medskip

\noindent \textbf{3. All possible alignments when length = 4}
\begin{verbatim}
    A--G    A-G-    AG--    A--G    -A-G    -AG-
    ACG-    AC-G    A-CG    -ACG    ACG-    AC-G

    -AG-    A-G-    --AG    --AG    -A-G    AG--
    A-CG    -ACG    ACG-    AC-G    A-CG    -ACG
\end{verbatim}
\medskip

\noindent \textbf{4. All possible alignments when length = 3}
\begin{verbatim}
    -AG    A-G    AG-
    ACG    ACG    ACG
\end{verbatim}
\medskip

\noindent \textbf{5. Alignment with the best score}
\begin{verbatim}
    ACG
    A-G	
\end{verbatim}

Score: 1

%
% Search space size of the brute-force approach
%
\subsubsection*{Search space size of the brute-force approach}
The search space size is the number of all possible alignments. It is 25 (10 + 12 + 3) for the example above. \\

\noindent \textbf{Rapid growth of search space size}

\begin{multicols}{2}
\begin{verbatim}
Example 1
    q: ACGACG, d: AGAG
\end{verbatim}
Search space size: 1289

\begin{verbatim}
Example 2
    q: ACGACGACGACG, d: AGAGAGAG
\end{verbatim}
Search space size: 4,673,345

\end{multicols}

%
% Exercise \thesection.2
%
\subsubsection*{Exercise \thesection.2}
Find the alignment with the best score for the sequences.  Use the simple scoring scheme below.

\begin{multicols}{2}
Sequences:
\begin{verbatim}
    q: A, d: AC
\end{verbatim}
\vfill\null
\columnbreak

\noindent Scoring scheme: \\ 
\null \quad $R_{ab}$ = 1 for a = b \\ 
\null \quad $R_{ab}$ = 0 for a $\neq$ b \\ 
\null \quad g = 1

\end{multicols} 

\begin{enumerate}
\item What are the maximum and minimum lengths of the alignment?
\item What is the search space size when the brute-force approach is used?
\item Identify all possible alignments.
\item What is the best score?
\end{enumerate}

%\end{document}
