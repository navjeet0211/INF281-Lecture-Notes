%\documentclass[12pt]{article}
%\usepackage[a4paper, margin=1in]{geometry} 
%\usepackage{graphicx} 
%\usepackage{hyperref}
%\usepackage{float}
%\usepackage{multicol}
%\usepackage{multirow}
%\usepackage{amsmath}
%\usepackage[font=small, labelfont=bf]{caption}
%
%\begin{document}

%
% Sequence patterns 
%
\subsection{Sequence patterns}
A sequence pattern can be used to find protein family members. The concept of finding patterns is similar to creating regular expressions.

%
% The PROSITE language
%
\subsubsection*{The PROSITE language}
\begin{itemize}
\item x: An arbitrary amino acid
\item -: Separating elements
\item \verb|[]|: A list of amino acids
\item \{\}: A list of not accepted amino acids
\item (): A range of en element
\end{itemize}

%
% Example of PROSITE
%
\subsubsection*{Example of PROSITE}
Find all matched sequences for the patterns. Assume the alphabet \verb|M = {A, G, C, T}|. \\

\noindent
Pattern 1: \verb|A - [GC] - {AGC}|
\begin{verbatim}
   AGT
   ACT
\end{verbatim}

\noindent
Pattern 2: \verb|A - x(1,3) - G|
\begin{verbatim}
   AxG
   AxxG
   AxxxG
\end{verbatim}

%
% Exercise \thesection.1
%
\subsubsection*{Exercise \thesection.1}
Find all matched sequences for the pattern. Assume the alphabet \verb|M = {A, G, C, T}|. \\

\noindent
Pattern: \verb|[AC] - {GCT} - x(1, 2) - T|

\bigskip 

\bigskip 

\bigskip 

%\end{document}
