%\documentclass[12pt]{article}
%\usepackage[a4paper, margin=1in]{geometry} 
%\usepackage{graphicx} 
%\usepackage{hyperref}
%\usepackage{float}
%\usepackage{multicol}
%\usepackage[font=small, labelfont=bf]{caption}
%
%\begin{document}

%
% Extension of gap penalties
%
\subsection{Extension of gap penalties}

%
% Types of gap penalties
%
\subsubsection*{Types of gap penalties}
Three types of gap penalties can be considered when creating an alignment. They treat a gap penalty differently depending on the gap length.

\begin{itemize}
\item Linear
\item Affine
\item Constant
\end{itemize}

%
% Gap penalty notation
%
\subsubsection*{Gap penalty notation}
\begin{itemize}
\item $g$: single gap penalty
\item $l$: length of a gap
\item $g_l$: gap penalty of length l
\item $g_{open}$: initial gap penalty
\item $g_{extend}$: extended gap penalty
\end{itemize}

%
% Linear gap penalty
%
\subsubsection*{Linear gap penalty}
It is the same as our simple scoring scheme. It treats a gap with multiple blanks as a result of several mutations. A gap of length $l$ can be calculated as: $g_l = g * l$.
\medskip 

\noindent
\textbf{Example of a gap of length 2}
\begin{verbatim}
    q: ACCCGT
    d: AC--GT
\end{verbatim}
The score of the gap (only the gap part) is 10 when $g$ = 5. 

%
% Affine gap penalty
%
\subsubsection*{Affine gap penalty}
It treats a gap with multiple blanks as a result of a single mutation. A gap with length $l$ can be calculated as: $g_l = g_{open} + (l – 1) * g_{extend}$.
\medskip 

\noindent
\textbf{Example of a gap of length 2}
\begin{verbatim}
    q: ACCCGT
    d: AC--GT
\end{verbatim}
The score of the gap (only the gap part) is 5.5 when $g_{open}$ and $g_{extend}$ are 5 and 0.5 respectively. 

%
% Constant gap penalty
%
\subsubsection*{Constant gap penalty}
It is similar to the affine gap penalty, but the score is independent form the gap length. A gap with length l can be calculated as: $g_l = g$ 
\medskip 

\noindent
\textbf{Example of a gap of length 2}
\begin{verbatim}
    q: ACCCGT
    d: AC--GT
\end{verbatim}
The score of the gap (only the gap part) for the alignment above is 5 when $g$ = 5. 

%
% Exercise \thesection.2
%
\subsubsection*{Exercise \thesection.2}
Calculate all three types of gap penalties for the gap in alignment 1 \& 2.

\begin{itemize}
\item $g$: 5
\item $g_{open}$: 5
\item $g_{extend}$: 0.5
\end{itemize}

\begin{multicols}{2}
\begin{verbatim}
Alignment 1
    q: CCCGG 
    d: CC-CG
\end{verbatim}

\begin{verbatim}
Alignment 2 
    q: CCCGG
    d: C---G
\end{verbatim}
\end{multicols}

\bigskip 

%\end{document}
