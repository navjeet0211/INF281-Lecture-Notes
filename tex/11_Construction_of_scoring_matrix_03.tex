%\documentclass[12pt]{article}
%\usepackage[a4paper, margin=1in]{geometry} 
%\usepackage{graphicx} 
%\usepackage{hyperref}
%\usepackage{float}
%\usepackage{multicol}
%\usepackage{multirow}
%\usepackage{amsmath}
%\usepackage{colortbl}
%\definecolor{lightgray}{RGB}{211, 211, 211}
%\usepackage[font=small, labelfont=bf]{caption}
%
%\begin{document}

%
% PAM – substitution matrix
%
\subsection{PAM – substitution matrix}
PAM is based on a substitution matrix created from experimental data.

%
% Relative mutability
%
\subsubsection*{Relative mutability}
The probabilities of amino acid mutations are calculated based on relative mutability. \\

$m_a : \dfrac{1}{100p_a} \times \dfrac{f_a}{f}$

%
% Example of relative mutability calculation
%
\subsubsection*{Example of relative mutability calculation}

\begin{itemize}

\item Frequencies of estimated mutations

\begin{table}[H]
\centering
\begin{tabular}{|rl|rl|rl|rl|}
\hline
$f_A$: & 2  & $f_G$: & 2 & $f_C$: & 4 & $f_T$: & 2  \\ \arrayrulecolor{lightgray} \hline \arrayrulecolor{black} 
$f:$ & 10 &   &   &  &   &  &   \\ \hline
\end{tabular}
\end{table}

\item Background frequencies

\begin{table}[H]
\centering
\begin{tabular}{|rl|rl|rl|rl|}
\hline
$p_A$: & 3/12  & $p_G$: & 4/12 & $p_C$: & 2/12 & $p_T$: & 3/12 \\ \arrayrulecolor{lightgray} \hline \arrayrulecolor{black} 
$100p_A$: & 23  & $100p_G$:  & 33.33  & $100p_C$:  & 16.67 & $100p_T$: & 25   \\ \hline
\end{tabular}
\end{table}

\item Relative mutability (1 PAM)

\begin{table}[H]
\centering
\begin{tabular}{|rl|rl|rl|rl|}
\hline
$m_A$: & 0.008  & $m_G$: & 0.006 & $m_C$: & 0.024 & $m_T$: & 0.008 \\ \hline
\end{tabular}
\end{table}

\end{itemize}

%
% Mutation probability
%
\subsubsection*{Mutation probability}
Mutation probabilities are summarized in a matrix format called substitution matrix.

\[
M_{ab} : m_{a} \times \dfrac{f_{ab}}{f_a} \quad M_{aa} : 1 - m_{a}
\]

%
% Example of substitution matrix
%
\subsubsection*{Example of substitution matrix}
\begin{itemize}
\item Frequencies of estimated mutations

\begin{table}[H]
\centering
\begin{tabular}{|rl|rl|rl|rl|}
\hline
$f_{AC}$: & 1  & $f_{AT}$: & 1 & $f_{GC}$: & 2 & $f_{CT}$: & 1  \\ \arrayrulecolor{lightgray} \hline \arrayrulecolor{black} 
$f_{CA}$: & 1  & $f_{TA}$: & 1 & $f_{CG}$: & 2 & $f_{TC}$: & 1  \\ \arrayrulecolor{lightgray} \hline \arrayrulecolor{black} 
$f_{A}$: & 2  & $f_{G}$: & 2 & $f_{C}$: & 4 & $f_{T}$: & 2 \\ \hline
\end{tabular}
\end{table}

\item Relative mutability (1 PAM)

\begin{table}[H]
\centering
\begin{tabular}{|rl|rl|rl|rl|}
\hline
$m_{A}$: & 0.008  & $m_{G}$: & 0.006 & $m_{C}$: & 0.024 & $m_{T}$: & 0.008 \\ \hline
\end{tabular}
\end{table}

\item Mutation probabilities

\begin{table}[H]
\centering
\begin{tabular}{|rl|rl|rl|rl|}
\hline
$m_{AC}$: & 0.004  & $m_{AT}$: & 0.004 &  &  &  &  \\ \arrayrulecolor{lightgray} \hline \arrayrulecolor{black} 
$m_{GC}$: & 0.006  & &  & &  &  &   \\ \arrayrulecolor{lightgray} \hline \arrayrulecolor{black} 
$m_{CA}$: & 0.006  & $m_{GC}$: & 0.012 & $m_{CT}$: & 0.006 &  &  \\ \arrayrulecolor{lightgray} \hline \arrayrulecolor{black} 
$m_{TA}$: & 0.004  & $m_{TC}$: & 0.004 &  &  &  &   \\ \arrayrulecolor{lightgray} \hline \arrayrulecolor{black} 
$m_{AA}$: & 0.992  & $m_{GC}$: & 0.994 & $m_{CC}$: & 0.976 & $m_{TT}$: & 0.992 \\ \hline
\end{tabular}
\end{table}

\item Substitution matrix

\begin{table}[H]
\centering
\begin{tabular}{l|llll}
  & A     & G     & C     & T     \\ \hline
A & 0.992 &       & 0.004 & 0.004 \\
G &       & 0.994 & 0.006 &       \\
C & 0.006 & 0.012 & 0.976 & 0.006 \\
T & 0.004 &       & 0.004 & 0.992
\end{tabular}
\end{table}

\end{itemize}

%
% Matrices for general evolutionary time
%
\subsubsection*{Matrices for general evolutionary time}
Markov chains can be used to generalize PAM with arbitrary τ values. For instance, the substitution value for 2 PAM (τ=2) for amino acids a to b can be calculated as:

\[
M_{ab}^2 = M_{ab} M_{bb} + M_{aa} M_{ab} + \sum_{c \notin \{a, b\}}{} M_{ac} M_{cb} = \sum_{c \in M}{} M_{ac} M_{cb}
\]

%
% Odds matrix
%
\subsubsection*{Odds matrix}
Substitution scores can be transformed to odds values. Odds values $O_{ab}$ are equal to $O_{ba}$, and therefore an odds matrix is symmetrical.

\[
O_{ab} = \dfrac{M_{ab}}{p_b}
\]

when $a \neq b$:

\[
O_{ab} 
= \dfrac{M_{ab}}{p_b} 
= m_a \times \dfrac{f_{ab}}{f_a} \times \dfrac{1}{p_b} 
= \dfrac{1}{100p_a} \times \dfrac{f_{a}}{f}  \times \dfrac{f_{ab}}{f_a} \times \dfrac{1}{p_b} 
= \dfrac{f_{ab}}{100fp_ap_b}
\]

%
% Transformation of an odds matrix to a score matrix
%
\subsubsection*{Transformation of an odds matrix to a score matrix}
Odds values can be further transformed to log-odds values.

\[
R_{ab} = \log O_{ab} = \log \dfrac{M_{ab}}{p_b}
\]

\bigskip 

%\end{document}
