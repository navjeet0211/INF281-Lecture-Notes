%\documentclass[12pt]{article}
%\usepackage[a4paper, margin=1in]{geometry} 
%\usepackage{graphicx} 
%\usepackage{hyperref}
%\usepackage{float}
%\usepackage{multicol}
%\usepackage{multirow}
%\usepackage[font=small, labelfont=bf]{caption}
%
%\begin{document} 

%
% N-gram based search
%
\subsection{N-gram based search}
Using n-grams is a useful method to find segment pairs.

%
% Equivalent or related concepts to n-gram
%
\subsubsection*{Equivalent or related concepts to n-gram} 
\begin{itemize}
\item q-gram
\item n-letter word
\item n-tuple
\item n-mer
\end{itemize}

%
% Create n-grams
%
\subsubsection*{Create n-grams} 
Decomposing a given sequence into n-letter words creates a list of n-grams.

\medskip 

\noindent
\textbf{Example}

\begin{verbatim}
    q: ACGATT 
    
    Word size: 2
        AC, CG, GA, AT, TT
        
    Word size: 3    
        ACG, CGA, GAT, ATT
        
\end{verbatim}

%
% Find segment pairs in database sequences
%
\subsubsection*{Find segment pairs in database sequences} 
N-grams can be used to find segment pairs.

\medskip

\noindent
\textbf{Example}

\begin{verbatim}
    q: ACGATT 
    2-gram: AC, CG, GA, AT, TT
    
    d1: CTAAG
    0 hit

    d2: CGTAT
    2 hits

    d3: ATAGA
    2 hits
\end{verbatim}

\bigskip 

%\end{document}
