%\documentclass[12pt]{article}
%\usepackage[a4paper, margin=1in]{geometry} 
%\usepackage{graphicx} 
%\usepackage{hyperref}
%\usepackage{float}
%\usepackage{multicol}
%\usepackage[font=small, labelfont=bf]{caption}
%
%\begin{document}

%
% Pairwise alignment
%
\subsection{Pairwise alignment}
A pairwise alignment is a basic sequence structure that consists of two sequences. A global alignment stretches to the whole part of two sequences, whereas a local alignment usually contains only part of the sequences.

%
% Pairwise alignment
%
\subsubsection*{Components of pairwise alignment}

We name two sequences as ‘database’ or ‘d’ and ‘query’ or ‘q’ through this course. They may represent sequences from two different species or organisms.
\\

\noindent
Identical sequences.
\begin{verbatim}
    q: ACGT
    d: ACGT
\end{verbatim}

\noindent
One mismatch.
\begin{verbatim}
    q: ACGT
    d: ACGA
\end{verbatim}

\noindent
The '-' symbol represents a blank. A single or a set of multiple blanks further represents a gap, which is an indication of insertion or deletion in the course of evolution between two organisms.
\begin{verbatim}
    q: ACGT
    d: A-GT
\end{verbatim}

\noindent
\textbf{N.B.} A gap cannot be aligned with another gap.

%
% Example of a simple scoring scheme
%
\subsubsection*{Example of a simple scoring scheme}
\begin{itemize}
\item Match: 1
\item Mismatch: 0
\item Gap penalty: 1 (use -1 for the actual calculation)
\end{itemize}

\noindent
We may use the following notation.
\begin{itemize}
\item $R_{ab}$ = 1 for a = b
\item $R_{ab}$ = 0 for a $\neq$ b
\item g = 1
\end{itemize}

%
% NEW PAGE
%
\newpage 

%
% Exercise \thesection.1
%
\subsubsection*{Exercise \thesection.1}
Use the simple scoring scheme above and calculate the scores of the following two alignments.

\begin{multicols}{2}
\begin{verbatim}
Alignment 1
    q: GCA-GCA
    d: GA-TG-A	
\end{verbatim}

\begin{verbatim}
Alignment 2 
    q: GCA-GCA
    d: GA-TG-A	
\end{verbatim}
\end{multicols}

%\end{document}
